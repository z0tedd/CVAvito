
\documentclass[a4paper,12pt]{article}
\usepackage{geometry}
\geometry{top=20mm, bottom=20mm, left=20mm, right=20mm}
\usepackage{enumitem}
\usepackage{titlesec}
\usepackage{setspace}
\usepackage{hyperref}
\usepackage{titlesec}
%%% Работа с русским языком
\titleformat{\section}{\normalfont\bfseries}{\thechapter}{14pt}{\bfseries}
\usepackage{cmap}					% поиск в PDF
\usepackage{mathtext} 				% русские буквы в фомулах
\usepackage[T2A]{fontenc}			% кодировка
\usepackage[utf8]{inputenc}			% кодировка исходного текста
\usepackage[english,russian]{babel}	% локализация и переносы
\usepackage{ fancyhdr} % улучшенная нумерация страниц
%%% Дополнительная работа с математикой
\usepackage{amsmath,amsfonts,amssymb,amsthm,mathtools} % AMS
\usepackage{icomma} % "Умная" запятая: $0,2$ --- число, $0, 2$ --- перечисление
%\renewcommand{\sfdefault}{cmss}
\usefont{T2A}{cmss}{m}{n}
%% Номера формул
\mathtoolsset{showonlyrefs=true} % Показывать номера только у тех формул, на которые есть \eqref{} в тексте.

%% Шрифты
\usepackage{euscript}	 % Шрифт Евклид
\usepackage{mathrsfs} % Красивый матшрифт
%% Свои команды
\DeclareMathOperator{\sgn}{\mathop{sgn}}

%% Перенос знаков в формулах (по Львовскому)
\newcommand*{\hm}[1]{#1\nobreak\discretionary{}
	{\hbox{$\mathsurround=0pt #1$}}{}}

%\renewcommand{\sfdefault}{cmss}
\usepackage{tempora}


\hypersetup{
    colorlinks=true,
    linkcolor=blue,
    urlcolor=blue
}

% Настройка разделов
\titleformat{\section}{\large\bfseries}{}{0em}{}
\titleformat{\subsection}{\normalsize\bfseries}{}{0em}{}

\begin{document}

\begin{center}
    \textbf{\LARGE Резюме}
\end{center}

\begin{center}
    \textbf{\large Файтельсон Антон Александрович} \\
    Телефон: +7 (910) 312-2108 \\
    Email: z0tedd@gmail.com \\

    GitHub: \url{https://github.com/z0tedd}
\end{center}

\vspace{10pt}

\section{Желаемая позиция}
\textbf{Middle Go-разработчик}

\section{Профессиональный опыт}

\subsection*{ООО «Desmax» (2021-2022)}
\textbf{Роль:} Junior Go-разработчик \\
\textbf{Описание компании:} Разработка программного обеспечения для автоматизации бухгалтерского учета. \\
\textbf{Обязанности:}
\begin{itemize}[noitemsep]
    \item Разработка и поддержка RESTful API для основных учетных модулей.
    \item Реализация модулей для обработки и хранения финансовых данных с использованием баз данных PostgreSQL.
    \item Оптимизация кода и рефакторинг существующих сервисов для повышения производительности.
    \item Участие в ревью кода, тестировании и дебаггинге приложений.
    \item Документирование API и написание базовых тестов на unit-тестирование.
\end{itemize}

\subsection*{Студия разработки "Дракон и Молния", Курск (2022-2023)}
\textbf{Роль:} Go-разработчик \\
\textbf{Описание компании:} Разработка и поддержка высоконагруженных сайтов и веб-сервисов для заказчиков. \\
\textbf{Обязанности:}
\begin{itemize}[noitemsep]
    \item Разработка backend-части веб-приложений на Go и поддержка микросервисной архитектуры.
    \item Интеграция сторонних API и взаимодействие с различными веб-сервисами.
    \item Создание и внедрение системы логирования и мониторинга сервисов.
    \item Настройка CI/CD процессов для автоматизации сборок и развертывания.
    \item Написание интеграционных и юнит-тестов для обеспечения качества кода.
\end{itemize}

\subsection*{Стартап «TextForEveryone» (2022-2023)}
\textbf{Роль:} Go-разработчик \\
\textbf{Описание компании:} Стартап, предоставляющий услуги speech-to-text для автоматизации обработки голосовых данных. \\
\textbf{Обязанности:}
\begin{itemize}[noitemsep]
    \item Разработка серверных модулей для обработки и преобразования аудиоданных в текст.
    \item Внедрение алгоритмов машинного обучения для распознавания речи и оптимизации работы сервиса.
    \item Участие в разработке микросервисной архитектуры для масштабирования решений.
    \item Работа с Docker и Kubernetes для контейнеризации и управления сервисами.
    \item Повышение производительности сервисов и снижение задержек в системе.
\end{itemize}

\section{Образование}
\textbf{Бакалавр компьютерных наук} \\
Курский государственный технический университет, 2023

\section{Навыки}
\textbf{Языки программирования:} Go, SQL, JavaScript \\
\textbf{Технологии и инструменты:} Docker, Kubernetes, REST API, gRPC, Git, PostgreSQL, Redis \\
\textbf{Методологии разработки:} Agile, Scrum \\
\textbf{Среды разработки:} Visual Studio Code, IntelliJ IDEA \\
\textbf{Операционные системы:} Linux, macOS, Windows

\section{Технологический стек}
\begin{itemize}[noitemsep]
    \item \textbf{Golang:} Разработка высокопроизводительных серверных приложений и микросервисов.
    \item \textbf{Docker и Kubernetes:} Создание и управление контейнерами для развёртывания и масштабирования приложений.
    \item \textbf{PostgreSQL и Redis:} Работа с базами данных, оптимизация запросов, хранение кэша.
    \item \textbf{gRPC и REST:} Построение API, реализация сервисов для межсервисного взаимодействия.
    \item \textbf{CI/CD:} Настройка процессов CI/CD (Jenkins, GitLab CI) для автоматизации разработки и выпуска.
    \item \textbf{Тестирование:} Написание юнит и интеграционных тестов с использованием Go testing framework.
\end{itemize}

\section{Дополнительная информация}
\begin{itemize}[noitemsep]
    \item Уверенные навыки работы с распределёнными системами.
    \item Опыт работы в высоконагруженных проектах и оптимизации серверного кода.
    \item Готовность к обучению и внедрению новых технологий в процессе работы.
\end{itemize}



\end{document}
